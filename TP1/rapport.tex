\documentclass[12pt, a4paper]{article}
\usepackage[francais]{babel}
\usepackage{caption}
\usepackage{graphicx}
\usepackage[T1]{fontenc}
\usepackage{listings}
\usepackage{geometry}
\usepackage[colorlinks=true,linkcolor=black,anchorcolor=black,citecolor=black,filecolor=black,menucolor=black,runcolor=black,urlcolor=black]{hyperref}

% \usepackage{mathpazo} --> Police à utiliser lors de rapports plus sérieux

\usepackage{fancyhdr}
\pagestyle{fancy}
\lhead{}
\rhead{}
\chead{}
\rfoot{\thepage}
\lfoot{Martin Baumgaertner}
\cfoot{}

\renewcommand{\headrulewidth}{0.4pt}
\renewcommand{\footrulewidth}{0.4pt}

\begin{document}
\begin{titlepage}
	\newcommand{\HRule}{\rule{\linewidth}{0.5mm}} 
	\center 
	\textsc{\LARGE iut de colmar}\\[6.5cm] 
	\textsc{\Large R401 -- Infrastructure de sécurité}\\[0.5cm] 
	\textsc{\large Année 2022-23}\\[0.5cm]
	\HRule\\[0.75cm]
	{\huge\bfseries TP1 - Chiffrement et PKI}\\[0.4cm]
	\HRule\\[1.5cm]
	\textsc{\large martin baumgaertner}\\[6.5cm] 

	\vfill\vfill\vfill
	{\large\today} 
	\vfill
\end{titlepage}
\newpage
\tableofcontents
\newpage
\section{Chiffrement}
    \subsection{Permutation}
    \subsubsection{Question 1}
    Le message déchiffré que j'obtiens est \texttt{MAITRE CORBEAU SUR UN ARBRE PERCHEX}
    Voici ce que j'ai fait pour obtenir ce message :
    \begin{enumerate}
        \item J'ai créé une liste des lettres de l'alphabet, dans l'ordre.
        \item J'ai créé une deuxième liste, contenant les nombres 1 à 6.
        \item J'ai apparié les lettres et les chiffres des deux listes, dans l'ordre spécifié dans la clé.
        \item J'ai utilisé les chiffres pour créer une nouvelle séquence de lettres, en remplaçant chaque lettre du message d'origine par la lettre qui lui correspond dans la nouvelle séquence.
        \item J'ai supprimé tous les espaces de la chaîne obtenue.
    \end{enumerate}

    \subsection{Substitution}
    \subsubsection{Question 2}
    Voici le message que j'ai trouvé \texttt{UN SECRET}

    \subsection{Diffie-Hellman}
    \subsubsection{Question 3}
    J'ai obtenu \texttt{K = 117}

    \subsection{Chiffrement symétrique AES}
    \subsubsection{Question 5}
    Voici le résultat que l'on obtient en base64 : \\

    \texttt{U2FsdGVkX1+QhnAugHjdc6bhMEMibXxWGQSE6ZKN56o=,}\\

    L’option “-nosalt” lors d’une commande openssl permet de ne pas rajouter de valeur aléatoire au code avant son hachage. En effet “salt” rajoute une valeur aléatoire pour renforcer le code/mot de passe.

    \newpage
    \section{PKI}   
    \subsection{Création du Certificat Racine (ou CA) et de sa clé privée.}
    \textbf{CA} signigie \textbf{C}ertificate \textbf{A}uthority. C'est une entité qui émet des certificats numériques.\\

    \subsection{Lecture du certificat}
    Ci-dessous, un tableau avec les différents \textbf{champs}
    et \textbf{fonctions} du certificat.
    \begin{center}
        \begin{tabular}{|c|c|}
            \hline
            \textbf{Champ} & \textbf{Fonction} \\
            \hline
            Version & Version du certificat \\
            \hline
            Serial Number & Numéro de série du certificat \\
            \hline
            Signature Algorithm & Algorithme de signature \\
            \hline
            Issuer & Entité qui a signé le certificat \\
            \hline
            Validity & Période de validité du certificat \\
            \hline
            Subject & Entité à qui le certificat est destiné \\
            \hline
            Subject Public Key Info & Informations sur la clé publique \\
            \hline
            X509v3 extensions & Extensions du certificat \\
            \hline
        \end{tabular}
    \end{center}

    \subsection{Contenu du CSR}
    \textbf{CSR} signifie \textbf{C}ertificate \textbf{S}igning \textbf{R}equest. C'est une requête de signature de certificat.\\
    Voici ci-dessous un tableau listant les \textbf{champs} et \textbf{fonctions} du CSR.
    \begin{center}
        \begin{tabular}{|c|c|}
            \hline
            \textbf{Champ} & \textbf{Fonction} \\
            \hline
            Version & Version du CSR \\
            \hline
            Subject & Entité à qui le certificat est destiné \\
            \hline
            Subject Public Key Info & Informations sur la clé publique \\
            \hline
            X509v3 extensions & Extensions du certificat \\
            \hline
        \end{tabular}
    \end{center}

    \subsection{Création de la liste de révocation}
    \textbf{CRL} signifie \textbf{C}ertificate \textbf{R}evocation \textbf{L}ist. C'est une liste de révocation de certificats.\\




\end{document}